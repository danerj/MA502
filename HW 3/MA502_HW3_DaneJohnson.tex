\documentclass[11pt]{article}

\usepackage{amssymb}
\usepackage{amsmath}

\title{MA 502 Homework 3}
\author{Dane Johnson}

\begin{document}
\maketitle

\section*{Exercise 1}

Given two bases $\mathbb{B}_1 = \{v_1,...,v_n\}$ and $\mathbb{B}_2 = \{w_1,...,w_n\}$ for the vector space $V$ and a linear transformation $L : V \rightarrow V$, prove that
$$[L]_{\mathbb{B}_2 \rightarrow \mathbb{B}_1}[a]_{\mathbb{B}_2} = [\mathbb{B}_2 \rightarrow \mathbb{B}_1][L]_{\mathbb{B}_1 \rightarrow \mathbb{B}_2}[a]_{\mathbb{B}_1} \;.$$

First we have that $$[L]_{\mathbb{B}_1 \rightarrow \mathbb{B}_2}[a]_{\mathbb{B}_2} = [La]_{\mathbb{B}_1} \,.$$

Next we have
\begin{align*}
[\mathbb{B}_2 \rightarrow \mathbb{B}_1][L]_{\mathbb{B}_1 \rightarrow \mathbb{B}_2}[a]_{\mathbb{B}_1} &= [\mathbb{B}_2 \rightarrow \mathbb{B}_1][La]_{\mathbb{B}_2} \\
&= [I]_{\mathbb{B}_2 \rightarrow \mathbb{B}_1} [La]_{\mathbb{B}_2} \\
&= [I(La)]_{\mathbb{B}_1} = [La]_{\mathbb{B}_1} \;.
\end{align*}

Therefore, we conclude that 
$$[L]_{\mathbb{B}_2 \rightarrow \mathbb{B}_1}[a]_{\mathbb{B}_2} = [\mathbb{B}_2 \rightarrow \mathbb{B}_1][L]_{\mathbb{B}_1 \rightarrow \mathbb{B}_2}[a]_{\mathbb{B}_1} \;.$$

\section*{Exercise 2}

Consider the linear map $L : \mathbb{R}^3 \rightarrow \mathbb{R}^3$ represented in canonical coordinates by the matrix
$$[L]_{\mathcal{C} \rightarrow \mathcal{C}} = \begin{pmatrix}
1&2&3 \\ 4&0&4 \\ 2&1&3
\end{pmatrix} \;.$$

First we find the null space and range of $L$ using the transformation's matrix representation (since the matrix representation will provide all the information we need about the transformation to find these spaces). \\

The null space consists of vectors from $\mathbb{R}^3$ that will satisfy:

$$\begin{pmatrix}
1&2&3 \\ 4&0&4 \\ 2&1&3
\end{pmatrix} \begin{pmatrix}
x_1 \\ x_2 \\ x_3
\end{pmatrix} = \begin{pmatrix}
0 \\ 0 \\ 0
\end{pmatrix} \;.$$

For computation we use row reduction on the corresponding augmented matrix:

\begin{align*}
\begin{pmatrix} 1&2&3 & 0 \\ 4&0&4 &0\\ 2&1&3&0 \end{pmatrix} &\sim
\begin{pmatrix}
1&2&3 & 0 \\ -2&0&-2 &0\\ 2&1&3&0
\end{pmatrix}
\sim \begin{pmatrix}
1&2&3 & 0 \\ -2&0&-2 &0\\ 0&1&1&0
\end{pmatrix} \\
&\sim \begin{pmatrix}
1&2&3 & 0 \\ -1&0&-1 &0\\ 0&1&1&0
\end{pmatrix} \sim \begin{pmatrix}
1&2&3 & 0 \\ 0&2&2 &0\\ 0&1&1&0
\end{pmatrix} \\ &\sim \begin{pmatrix}
1&2&3 & 0 \\ 0&1&1 &0\\ 0&0&0&0
\end{pmatrix} \sim \begin{pmatrix}
1&0&1 & 0 \\ 0&1&1 &0\\ 0&0&0&0
\end{pmatrix}
\end{align*}
Translating the results of our matrix computation to the properties of $L$, we see that if $(x_1,x_2,x_3) \in \mathcal{N}(L)$, then $x_1 = -x_3$ and $x_2 = -x_3$, where we may choose $x_3$ at will. Thus the null space of $L$ is the set$$\mathcal{N}(L) = \{c(-1,-1,1) \in \mathbb{R}^3 \, , c \in \mathbb{R}\} \,.$$

To find the range of $L$, we first note that the columns of the matrix are dependent. In particular, 

$$\begin{pmatrix} 1\\4\\2 \end{pmatrix} + \begin{pmatrix} 2\\0\\1 \end{pmatrix} = \begin{pmatrix} 3\\4\\3 \end{pmatrix} \; .$$

Therefore the column space of the matrix is the same as the column space of the first two columns of the matrix. Since the column space of the matrix is a matrix representation of the range of the linear transformation $L$, we see that the range of $L$ is the set:

$$ \{c_1(1,4,2) + c_2(2,0,1) \; | \; c_1,c_2 \in \mathbb{R}\} \;.$$

Consider the linear system $Lv = (1,2,0)$. To see if a solution exists, we use matrix computations:

\begin{align*}
\begin{pmatrix} 1&2&3&1 \\ 4&0&4&2\\ 2&1&3&0 \end{pmatrix} &\sim \begin{pmatrix} 1&2&3&1 \\ 0&-8&-8&-2\\ 0&-3&-3&0 \end{pmatrix} \\
&\sim \begin{pmatrix} 1&2&3&1 \\ 0&0&0&-2\\ 0&1&1&0 \end{pmatrix} \sim \begin{pmatrix} 1&0&1&1 \\ 0&1&1&0\\ 0&0&0&-2 \end{pmatrix} \;.
\end{align*}

In order for $v = (v_1,v_2,v_3)$ to satisfy the linear system, this row reduced system tells us that $0v_3 = -2$, so that $0 = -2$. Since this cannot be, we conclude that there is no solution to the system $Lv = (1,2,0)$. \\

Consider the linear system $Lv = (6,8,6)$. We use row reduction to determine whether solutions to this system exist:

\begin{align*}
\begin{pmatrix} 1&2&3&6 \\ 4&0&4&8\\ 2&1&3&6 \end{pmatrix} &\sim \begin{pmatrix} 1&2&3&1 \\ 0&-8&-8&-16\\ 0&-3&-3&-6 \end{pmatrix} \\
&\sim \begin{pmatrix} 1&2&3&1 \\ 0&1&1&2\\ 0&-1&-1&-2 \end{pmatrix} \sim \begin{pmatrix} 1&2&3&6 \\ 0&1&1&2\\ 0&0&0&0 \end{pmatrix} \\
&\sim \begin{pmatrix} 1&0&1&2 \\ 0&1&1&2\\ 0&0&0&0 \end{pmatrix}
\end{align*}

From this we have that if $v = (v_1,v_2,v_3)$ is a solution to the system it must be the case that $v_1 = 2-v_3$ and $v_2 = 2-v_3$. This shows that any vector of the form
$$v = \begin{pmatrix} 2\\2\\0 \end{pmatrix} + c\begin{pmatrix}
-1\\-1\\1
\end{pmatrix} \; c \in \mathbb{R} \,.$$

Since we may choose $c$ freely, there are infinitely many solutions to the system. For one specific solution, set $c=0$ in the above to get $L(2,2,0) = (6,8,6)$. 

\section*{Exercise 3}

Consider the operator $T: \mathbb{P} \rightarrow \mathbb{P}$, $T(p) = \int p(x) \, dx $, where $\mathbb{P}$ is the space of all polynomials. We find the null space and range of $T$.\\

Note that if $p(x)$ is not the zero polynomial, then $\int p(x) \, dx \neq 0$ and that if $p(x)$ is the zero polynomial, then $\int p(x) \, dx = 0$. So the null space of $T$ includes the zero polynomial but no other polynomial. \\

In order for $\mathbb{P}$ to indeed be a vector space, we must work under the agreement that the constant of integration is 0. \\

The range of $T$ is the set of all polynomials in $\mathbb{P}$ with a zero constant term. That is, $p(x)$ of the form $p(x) = a_1x +...+a_nx^n$ for some natural number $n$. To see why, first we see that if $p(x)$ is of the form $p(x) = a_0 +a_1x+...+a_nx^n$ with $a_0 \neq 0$, then there does not exist any polyomial $q(x) \in \mathbb{P}$ such that $T(q) = \int q(x) \, dx = p(x)$ since integration of a polynomial cannot result in any constant terms unless the constant of integration is nonzero, and we assumed that the constant of integration is zero. Next, if $p(x) = a_1x+...+a_nx^n \in \mathbb{P}$, set $q(x) = a_1 + 2a_2x +...+na_nx^{n-1}$. Then $T(q) = \int (a_1 + 2a_2x +... na_nx^{n-1})\, dx = p(x)$.

\section*{Exercise 4}

Let $T: \mathbb{R}^3 \rightarrow \mathbb{R}^{100}$ be a linear transformation. We show that $T$ cannot be a surjective transformation. \\

In class we proved that if $T : X \rightarrow Y$ is a linear transformation between vectors spaces $X$ and $Y$ and if dim $Y >$ dim $X$, then $T$ is not surjective. Here we see that dim $\mathbb{R}^3 = 3$ and dim $\mathbb{R}^{100} = 100$. Since dim $\mathbb{R}^{100} >$ dim $\mathbb{R}^3$, we conclude that $T$ is not surjective. 


\section*{Exercise 5}

Let $T: \mathbb{R}^{100} \rightarrow \mathbb{R}^3$ be a linear transformation. We show that $T$ cannot be an injective transformation. \\

In class we proved that if $T : X \rightarrow Y$ is a linear transformation between vectors spaces $X$ and $Y$ and if dim $Y <$ dim $X$, then $T$ is not injective. Here we see that dim $\mathbb{R}^{100} = 100$ and dim $\mathbb{R}^3 = 3$. Since dim $\mathbb{R}^3 <$ dim $\mathbb{R}^{100}$, we conclude that $T$ is not injective. \\

It is possible for such a map to be onto. Take $T : \mathbb{R}^{100} \rightarrow \mathbb{R}^3$, $T((x_1,x_2,x_3,...,x_{100})) = (x_1,x_2,x_3)$. Then for any element of $\mathbb{R}^3$, we see that any vector from $\mathbb{R}^{100}$ for which the first three components are the same as the target vector will suffice (while $x_4,x_5,...,x_{100}$ may be taken arbitrarily). 

\end{document}